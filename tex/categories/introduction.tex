\PassOptionsToPackage{unicode=true}{hyperref} % options for packages loaded elsewhere
\PassOptionsToPackage{hyphens}{url}
%
\documentclass[10pt,b5paper,pandoc]{bxjsarticle}
\usepackage[]{cmbright}
\usepackage{amssymb,amsmath}
\usepackage{ifxetex,ifluatex}
\usepackage{fixltx2e} % provides \textsubscript
\ifnum 0\ifxetex 1\fi\ifluatex 1\fi=0 % if pdftex
  \usepackage[T1]{fontenc}
  \usepackage[utf8]{inputenc}
  \usepackage{textcomp} % provides euro and other symbols
\else % if luatex or xelatex
  \usepackage{unicode-math}
  \defaultfontfeatures{Ligatures=TeX,Scale=MatchLowercase}
\fi
% use upquote if available, for straight quotes in verbatim environments
\IfFileExists{upquote.sty}{\usepackage{upquote}}{}
% use microtype if available
\IfFileExists{microtype.sty}{%
\usepackage[]{microtype}
\UseMicrotypeSet[protrusion]{basicmath} % disable protrusion for tt fonts
}{}
\IfFileExists{parskip.sty}{%
\usepackage{parskip}
}{% else
\setlength{\parindent}{0pt}
\setlength{\parskip}{6pt plus 2pt minus 1pt}
}
\usepackage{hyperref}
\hypersetup{
            pdftitle={圏論},
            pdfauthor={M. Nakata},
            pdfborder={0 0 0},
            breaklinks=true}
\urlstyle{same}  % don't use monospace font for urls
\setlength{\emergencystretch}{3em}  % prevent overfull lines
\providecommand{\tightlist}{%
  \setlength{\itemsep}{0pt}\setlength{\parskip}{0pt}}
\setcounter{secnumdepth}{5}
% Redefines (sub)paragraphs to behave more like sections
\ifx\paragraph\undefined\else
\let\oldparagraph\paragraph
\renewcommand{\paragraph}[1]{\oldparagraph{#1}\mbox{}}
\fi
\ifx\subparagraph\undefined\else
\let\oldsubparagraph\subparagraph
\renewcommand{\subparagraph}[1]{\oldsubparagraph{#1}\mbox{}}
\fi

% set default figure placement to htbp
\makeatletter
\def\fps@figure{htbp}
\makeatother


\title{圏論}
\author{M. Nakata}
\date{}

\def\theparagraph{}
\def\thesubparagraph{}
\def\lt{<}
\def\gt{>}
\begin{document}
\maketitle

{
\setcounter{tocdepth}{3}
\tableofcontents
}
\hypertarget{ux53c2ux8003ux6587ux732e}{%
\section{参考文献}\label{ux53c2ux8003ux6587ux732e}}

\begin{itemize}
\tightlist
\item
  Rotman. \emph{An Introduction to Homological Algebra}. Springer.
\item
  斎藤恭司,土岡俊介『ベーシック圏論
  普遍性からの速習コース』(丸善出版)
\end{itemize}

\hypertarget{topology-introduction}{%
\section{導入}\label{topology-introduction}}

圏論とは,群,環,ベクトル空間,集合,位相空間などの数学的構造を一般化したものである.今ではその応用の範囲は広く及ぶ.

\hypertarget{ux570f}{%
\subsection{圏}\label{ux570f}}

\hypertarget{ux5b9aux7fa9}{%
\paragraph{定義}\label{ux5b9aux7fa9}}
\addcontentsline{toc}{paragraph}{定義}

\textbf{圏}(\emph{category})\(\mathscr C\)
は,以下の要素から構成される:

\begin{enumerate}
\def\labelenumi{\roman{enumi})}
\tightlist
\item
  \textbf{対象}(\emph{object})とよばれるものからなる類
  \(\mathrm{obj}\,\mathscr C\),
\item
  各対象 \(A, B \in \mathrm{obj}\,\mathscr C\) に対し,集合
  \(\mathrm{Hom}\,(A, B)\) が対応する.\(\mathrm{Hom}\,(A,B)\) の元
  \(f \in \mathrm{Hom}\,(A,B)\) を \(A\) から \(B\)
  への\textbf{射}(\emph{morphism})といい,\(f \colon A \to B\)
  あるいは \(A \xrightarrow{f} B\) と表す.
\end{enumerate}

さらに,次の性質をみたす:

\begin{enumerate}
\def\labelenumi{\arabic{enumi})}
\tightlist
\item
  各対象 \(A, B, A', B' \in \mathrm{obj}\,\mathscr C\)
  に対し,\(A \neq A'\) または \(B \neq B'\) ならば
  \(\mathrm{Hom}\,(A, B) \cap \mathrm{Hom}\,(A', B') = \emptyset\)
  である.
\item
  \(\mathscr C\) における2つの射 \(f \colon A \to B\) と
  \(g \colon B \to C\) に対し,\(f\) と \(g\)
  の\textbf{合成射}(\emph{composite})とよばれる射
  \(gf \colon A \to C\)
  が存在し結合律をみたす.すなわち,\(A \xrightarrow f B \xrightarrow g C \xrightarrow h D\)
  のとき \(h(gf) = (hg)f\) が成立つ.
\item
  各対象 \(A \in \mathrm{obj}\,\mathscr C\)
  に対し\textbf{恒等射}(\emph{identity})とよばれる射
  \(1_A \colon A \to A\) が存在し,\(f \colon A \to B\) ならば
  \(f1_A = f\) および \(1_ B f = f\) が成立つ.
\end{enumerate}

\(\mathrm{obj}\,\mathscr C\) が小さい類のとき,すなわち
\(\mathrm{obj}\,\mathscr C\) が集合のとき圏 \(\mathscr C\)
は\textbf{小さい}(\emph{small})という.

\hypertarget{ux4f8b}{%
\paragraph{例}\label{ux4f8b}}
\addcontentsline{toc}{paragraph}{例}

\end{document}
