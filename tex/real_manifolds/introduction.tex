\PassOptionsToPackage{unicode=true}{hyperref} % options for packages loaded elsewhere
\PassOptionsToPackage{hyphens}{url}
%
\documentclass[10pt,b5paper,pandoc]{bxjsarticle}
\usepackage[]{cmbright}
\usepackage{amssymb,amsmath}
\usepackage{ifxetex,ifluatex}
\usepackage{fixltx2e} % provides \textsubscript
\ifnum 0\ifxetex 1\fi\ifluatex 1\fi=0 % if pdftex
  \usepackage[T1]{fontenc}
  \usepackage[utf8]{inputenc}
  \usepackage{textcomp} % provides euro and other symbols
\else % if luatex or xelatex
  \usepackage{unicode-math}
  \defaultfontfeatures{Ligatures=TeX,Scale=MatchLowercase}
\fi
% use upquote if available, for straight quotes in verbatim environments
\IfFileExists{upquote.sty}{\usepackage{upquote}}{}
% use microtype if available
\IfFileExists{microtype.sty}{%
\usepackage[]{microtype}
\UseMicrotypeSet[protrusion]{basicmath} % disable protrusion for tt fonts
}{}
\IfFileExists{parskip.sty}{%
\usepackage{parskip}
}{% else
\setlength{\parindent}{0pt}
\setlength{\parskip}{6pt plus 2pt minus 1pt}
}
\usepackage{hyperref}
\hypersetup{
            pdftitle={多様体論},
            pdfauthor={M. Nakata},
            pdfborder={0 0 0},
            breaklinks=true}
\urlstyle{same}  % don't use monospace font for urls
\setlength{\emergencystretch}{3em}  % prevent overfull lines
\providecommand{\tightlist}{%
  \setlength{\itemsep}{0pt}\setlength{\parskip}{0pt}}
\setcounter{secnumdepth}{5}
% Redefines (sub)paragraphs to behave more like sections
\ifx\paragraph\undefined\else
\let\oldparagraph\paragraph
\renewcommand{\paragraph}[1]{\oldparagraph{#1}\mbox{}}
\fi
\ifx\subparagraph\undefined\else
\let\oldsubparagraph\subparagraph
\renewcommand{\subparagraph}[1]{\oldsubparagraph{#1}\mbox{}}
\fi

% set default figure placement to htbp
\makeatletter
\def\fps@figure{htbp}
\makeatother


\title{多様体論}
\author{M. Nakata}
\date{}

\def\theparagraph{}
\def\thesubparagraph{}
\def\lt{<}
\def\gt{>}
\begin{document}
\maketitle

{
\setcounter{tocdepth}{3}
\tableofcontents
}
\hypertarget{ux53c2ux8003ux6587ux732e}{%
\section{参考文献}\label{ux53c2ux8003ux6587ux732e}}

\begin{itemize}
\tightlist
\item
  Loring W. Tu. \emph{An Introduction to Manifolds}. Springer.
\item
  小林昭七『接続の微分幾何とゲージ理論』(裳華房)
\end{itemize}

\hypertarget{ux5c0eux5165}{%
\section{導入}\label{ux5c0eux5165}}

\hypertarget{ux30d9ux30afux30c8ux30ebux675f}{%
\subsection{ベクトル束}\label{ux30d9ux30afux30c8ux30ebux675f}}

\(M\) を \(n\) 次元多様体として,その点 \(p \in M\) における接空間を
\(T_pM\),余接空間を \(T^*_p M\) と書く.\((x^1, \dotsc, x^n)\) を点
\(p\) の近傍での局所座標系とすると, \(X \in T_pM\) に対して \[
X = \sum a^i \frac{\partial }{\partial x^i}
\] のとき \(X = (x^1(p), \dotsc, x^n(p), a^1, \dotsc, a^n)\)
として,\(TM = \bigcup_{p \in M} T_pM\)
に局所座標系を定めることができる.\(T^*M = \bigcup_{p \in M} T^*_p M\)
にも同様に局所座標系を定義できる.

\hypertarget{ux5b9aux7fa9}{%
\paragraph{定義}\label{ux5b9aux7fa9}}
\addcontentsline{toc}{paragraph}{定義}

多様体 \(M\) 上の\textbf{ベクトル束}(\emph{vector bundle})とは,多様体
\(E\) であって次の条件をみたすものである:

\begin{enumerate}
\def\labelenumi{\roman{enumi})}
\tightlist
\item
  微分可能な写像 \(\pi \colon E \to M\) があり,各点 \(x \in M\)
  に対して \(E_x = \pi^{-1}(x)\) は同じ次元 \(r\)
  のベクトル空間である.\(E_x\) を \(x\)
  上の\textbf{ファイバー}(\emph{fiber})という.
\item
  各点 \(x \in M\) とその近傍 \(U \subset M\) に対して,微分同相
  \(\varphi \colon \pi^{-1}(U) \to U \times \boldsymbol{R}^r\)
  が存在して,\(\{ y \} \subset M\) への制限が線形同型となる.
\end{enumerate}

\hypertarget{ux4f8b}{%
\paragraph{例}\label{ux4f8b}}
\addcontentsline{toc}{paragraph}{例}

\begin{enumerate}
\def\labelenumi{\roman{enumi})}
\tightlist
\item
  \(TM\) と \(T^*M\) はともにファイバーの次元が \(n\)
  のベクトル束である.実際,\(X \in T_pM\) に対して \(\pi(X) = p\)
  とすればこれはファイバーを与え,\(\pi^{-1}(p) = T_pM\) と \(R^n\)
  は線形同型となる.\(T^*M\) も同様.
\item
  \textbf{直積束}(\emph{product
  bundle})\(E = M \times \boldsymbol{R}^r\) はベクトル束である.
\end{enumerate}

\hypertarget{ux5b9aux7fa9-1}{%
\paragraph{定義}\label{ux5b9aux7fa9-1}}
\addcontentsline{toc}{paragraph}{定義}

\(\{ U_\alpha \} \subset M\) を \(M\) の被覆とすると各
\(\pi^{-1}(U_\alpha)\) は \(U_\alpha \times \boldsymbol{R}^r\)
と同型になる.よって,\(x \in U_\alpha \cap U_\beta\) に対して2つの同型
\[
\varphi_\alpha(x) \colon E_x \to \boldsymbol{R}^r\quad \varphi_\beta(x) \colon E_x \to \boldsymbol{R}^r
\]
が存在する.このとき,\(\psi_{\alpha\beta}(x) = \varphi_\alpha(x) \varphi_\beta(x)^{-1}\)
によって
\(\psi_{\alpha\beta} \colon U_\alpha \cap U_\beta \to GL(r, \boldsymbol{R})\)
が定まる.これらの写像の族 \(\{ \psi_{\alpha\beta} \}\) を被覆
\(\{ U_\alpha \}\) に対する \(E\) の\textbf{変換関数}(\emph{transition
functions})とよぶ.

\hypertarget{ux6ce8}{%
\paragraph{注}\label{ux6ce8}}
\addcontentsline{toc}{paragraph}{注}

\(E\) の変換関数 \(\{ \psi_{\alpha\beta} \}\) は次の性質をもつ:
\begin{align}
&\psi_{\alpha\beta}(x) \psi_{\beta\gamma}(x) = \psi_{\alpha\gamma}(x)\quad (x \in U_\alpha \cap U_\beta \cap U_\gamma);\\
&\psi_{\alpha\alpha}(x) = I_r;\\
&\psi_{\beta\alpha}(x) = \psi_{\alpha\beta}(x)^{-1}.
\end{align}

\end{document}
