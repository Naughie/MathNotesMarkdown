\PassOptionsToPackage{unicode=true}{hyperref} % options for packages loaded elsewhere
\PassOptionsToPackage{hyphens}{url}
%
\documentclass[10pt,b5paper,pandoc]{bxjsarticle}
\usepackage[]{cmbright}
\usepackage{amssymb,amsmath}
\usepackage{ifxetex,ifluatex}
\usepackage{fixltx2e} % provides \textsubscript
\ifnum 0\ifxetex 1\fi\ifluatex 1\fi=0 % if pdftex
  \usepackage[T1]{fontenc}
  \usepackage[utf8]{inputenc}
  \usepackage{textcomp} % provides euro and other symbols
\else % if luatex or xelatex
  \usepackage{unicode-math}
  \defaultfontfeatures{Ligatures=TeX,Scale=MatchLowercase}
\fi
% use upquote if available, for straight quotes in verbatim environments
\IfFileExists{upquote.sty}{\usepackage{upquote}}{}
% use microtype if available
\IfFileExists{microtype.sty}{%
\usepackage[]{microtype}
\UseMicrotypeSet[protrusion]{basicmath} % disable protrusion for tt fonts
}{}
\IfFileExists{parskip.sty}{%
\usepackage{parskip}
}{% else
\setlength{\parindent}{0pt}
\setlength{\parskip}{6pt plus 2pt minus 1pt}
}
\usepackage{hyperref}
\hypersetup{
            pdftitle={集合論},
            pdfauthor={M. Nakata},
            pdfborder={0 0 0},
            breaklinks=true}
\urlstyle{same}  % don't use monospace font for urls
\setlength{\emergencystretch}{3em}  % prevent overfull lines
\providecommand{\tightlist}{%
  \setlength{\itemsep}{0pt}\setlength{\parskip}{0pt}}
\setcounter{secnumdepth}{5}
% Redefines (sub)paragraphs to behave more like sections
\ifx\paragraph\undefined\else
\let\oldparagraph\paragraph
\renewcommand{\paragraph}[1]{\oldparagraph{#1}\mbox{}}
\fi
\ifx\subparagraph\undefined\else
\let\oldsubparagraph\subparagraph
\renewcommand{\subparagraph}[1]{\oldsubparagraph{#1}\mbox{}}
\fi

% set default figure placement to htbp
\makeatletter
\def\fps@figure{htbp}
\makeatother


\title{集合論}
\author{M. Nakata}
\date{}

\def\theparagraph{}
\def\thesubparagraph{}
\def\lt{<}
\def\gt{>}
\begin{document}
\maketitle

{
\setcounter{tocdepth}{3}
\tableofcontents
}
\hypertarget{ux53c2ux8003ux6587ux732e}{%
\section{参考文献}\label{ux53c2ux8003ux6587ux732e}}

\begin{itemize}
\tightlist
\item
  松坂和夫『集合・位相入門』(岩波書店)
\item
  内田伏一『数学シリーズ 集合と位相』(裳華房)
\item
  兒玉之宏,永見啓応『位相空間論』(岩波書店)
\end{itemize}

\hypertarget{sets-introduction}{%
\section{導入}\label{sets-introduction}}

ここでは,集合を厳密には定義しない.これを厳密に扱う分野は数学基礎論とよばれる.基礎論の内容は抽象的で難しいため,別に解説する.差し当たっては簡単に(ただし多くの教科書が採っている方法で)集合を定義した後,集合論を展開する.

\hypertarget{ux96c6ux5408ux306eux66f8ux304dux65b9}{%
\subsection{集合の書き方}\label{ux96c6ux5408ux306eux66f8ux304dux65b9}}

何らかのものの集まりを,\textbf{集合}(\emph{set})あるいは\textbf{族}(\emph{family}),\textbf{系}(\emph{system})という.集合を構成している個々のもののことをその集合の\textbf{元}(\emph{element},
\emph{member})あるいは\textbf{要素}とよぶ.\(x\) がある集合 \(X\)
の元であることを \(x \in X\) または \(X \ni x\) と表す.このとき,\(x\)
は \(X\) に\textbf{属する}(\emph{belongs
to})あるいは\textbf{含まれる}(\emph{is contained in})といい,また
\(X\) は \(x\) を\textbf{含む}(\emph{contains})という.\(x \in X\)
でないことを \(x \notin X\) と書く.

2つの集合 \(X\) と \(Y\) が与えられたとき,\(x \in X\) のときいつも
\(x \in Y\) でもあるとき,\(X\) は \(Y\)
の\textbf{部分集合}(\emph{subset})であるといい,\(X \subset Y\)
あるいは \(Y \supset X\) と書く.\(X \subset Y\) のとき,\(X\) は \(Y\)
に\textbf{含まれる}(\emph{is contained in}),あるいは \(Y\) は \(X\)
を\textbf{含む}(\emph{contains})ともいう.\(X \subset Y\) かつ
\(Y \subset X\) であるとき,\(X\) と \(Y\)
は\textbf{同じ集合}(\emph{identical sets})といい,\(X = Y\)
と表す.\(X \subset Y\) であるが \(X = Y\) でないとき,すなわち \(X\)
の元 \(x \in X\) であって \(x \notin Y\) なるものが存在するとき,\(X\)
は \(Y\) の\textbf{真部分集合}(\emph{proper
set})といい,\(X \subsetneq Y\) と書く.\(X\)
の元1つのみからなる部分集合 \(\{ x \} \subset X\)
を\textbf{シングルトン}(\emph{singleton})という. 2つの集合 \(X\) と
\(Y\) に対して,\(X\) または \(Y\) に属する元の集合を \(X\) と \(Y\)
の\textbf{和集合}(\emph{union})あるいは単に\textbf{和}(\emph{sum})といい,\(X\)
の元でもあり同時に \(Y\) の元でもあるようなものの集合を \(X\) と \(Y\)
の\textbf{共通部分}(\emph{intersection})という.和集合と共通部分をそれぞれ
\(X \cup Y\),\(X \cap Y\) で表す.\(X\) の元のうち \(Y\)
には属さない元の集合を \(Y\) の \(X\)
における\textbf{補集合}(\emph{complement})といい,\(X \setminus Y\)
と書く.文脈上明らかな場合は単に \(Y^c\) と書いて \(X\)
における補集合を表す.

集合 \(X\) の元がある条件 \(P\)
をみたすもの全体であるとき,\(X = \{ a \,:\,P \}\)
と表す.\(X = \{ a \mid P \}\)
と書くこともある.\(\alpha, \beta, \gamma, \dotsc\) 等で \(X\)
の元を代表して \(X = \{ \alpha, \beta, \gamma, \dotsc \}\)
などと書いてもよい.これによれば,和集合,共通部分,補集合はそれぞれ
\begin{align}
X \cup Y &= \{ a \,:\,a \in X \ \text{または} \ a \in Y \},\\
X \cap Y &= \{ a \,:\,a \in X \ \text{かつ}   \ a \in Y \},\\
X \setminus Y &= \{ a \,:\,a \in X \ \text{かつ} \ a \notin Y \}
\end{align} となる.集合 \(X \setminus X\) を \(\emptyset\)
と書き,\textbf{空集合}(\emph{empty set}, \emph{void set})とよぶ.

\hypertarget{ux4f8b}{%
\paragraph{例}\label{ux4f8b}}
\addcontentsline{toc}{paragraph}{例}

\begin{enumerate}
\def\labelenumi{\roman{enumi})}
\tightlist
\item
  \(\boldsymbol N = \{ 1, 2, 3, \dotsc \}\).自然数全体の集合.
\item
  \(\boldsymbol Z = \{ \dotsc, -3, -2, -1, 0, 1, 2, 3, \dotsc \}\).整数全体の集合.
\item
  \(\boldsymbol Q = \{ m / n \,:\,m \in \boldsymbol Z, \ n \in \boldsymbol N \}\).有理数全体の集合.
\item
  \(\boldsymbol R = \{ a \,:\,a \ \text{は実数}\}\).実数全体の集合.\(r \in \boldsymbol R\)
  に対して,\(\boldsymbol R_{>r}\) で \(r\)
  より大きい実数全体の集合を表す.\(\boldsymbol Q\) や \(\boldsymbol Z\)
  なども同様.
\item
  \(\boldsymbol C = \{ a + b \mathrm{i}\} \,:\,a, b \in \boldsymbol R \}\).複素数全体の集合.ただし,\(\mathrm{i}= \sqrt{-1}\)
  は虚数単位.
\item
  \(A = \{ 1, 2, 3\}, B = \{3, 0, \pi\}\)
  のとき,\(A \cup B = \{ 0, 1, 2, 3, \pi \},\ A \cap B = \{ 3 \},\ A \setminus B = \{ 1, 2 \}\).また,\(A \subset \boldsymbol N, B \subset \boldsymbol C\).
\item
  \(\boldsymbol N \subset \boldsymbol Z \subset \boldsymbol Q \subset \boldsymbol R \subset \boldsymbol C\).
\end{enumerate}

\(\boldsymbol N, \boldsymbol Z, \boldsymbol Q\)
は集合論によって定義できる.\(\boldsymbol Q\) から \(\boldsymbol R\)
を造り出す操作を\textbf{完備化}(\emph{completion})という.\(\boldsymbol R\)
から \(\boldsymbol C\) は,\textbf{代数拡大}(\emph{algebraic
extension})によって得られる.\(\Box\)

集合を要素とする集合を,\textbf{集合族}(\emph{family of
sets})あるいは\textbf{集合系}(\emph{system of
sets})とよぶ.集合族は,それに属する個々の集合を明示するために\textbf{インデックス}(\emph{index})を付して表されることが多い.たとえばインデックスの集合を
\(\Lambda\) としたとき,\(\{ A_\lambda \,:\,\lambda \in \Lambda \}\)
あるいは単に \(\{ A_\lambda \}\) のように書く.このとき \(\Lambda\)
を\textbf{添字集合}(\emph{index
set})といい,\(\{ A_\lambda \,:\,\lambda \in \Lambda \}\) を
\(\Lambda\) によって\textbf{添字付けられた集合族}(\emph{family of sets
indiced by \(\Lambda\)})という.集合族の各元 \(A_\lambda\)
がすべて共通の集合 \(X\) の部分集合のとき,すなわちすべての
\(A_\lambda\) に対して \(A_\lambda \subset X\)
のとき,\(\{ A_\lambda \,:\,\lambda \in \Lambda \}\) を \(X\)
の\textbf{部分集合族}(\emph{family of
subsets})とよぶ.任意の相異なるインデックス
\(\lambda, \mu \in \Lambda\) に対して
\(A_\lambda \cap A_\mu = \emptyset\) であるとき,集合族
\(\{ A_\lambda \,:\,\lambda \in \Lambda \}\)
は\textbf{互いに素}(\emph{pairwise disjoint})という.集合族
\(\{ A_\lambda \,:\,\lambda \in \Lambda \}\)
に対して,その\textbf{和集合}と\textbf{共通部分}を2つの集合の場合と同様に定義できる.すなわち,和集合とは少なくとも1つの
\(A_\lambda\) に属する元の集合,共通部分はすべての \(A_\lambda\)
に属する元の集合である.和集合は
\(\bigcup \{ A_\lambda \,:\,\lambda \in \Lambda \}\) または
\(\bigcup_{\lambda \in \Lambda} A_\lambda\) や単に \(\bigcup A_\lambda\)
と書く.共通部分も,\(\bigcap \{ A_\lambda \,:\,\lambda \in \Lambda \}\)
や \(\bigcap_{\lambda \in \Lambda} A_\lambda\),\(\bigcap A_\lambda\)
のように書く.\(\{ A_\lambda \,:\,\lambda \in \Lambda \}\)
が互いに素なとき,その和集合を\textbf{非交和}(\emph{disjoint
union})といい \(\coprod_{\lambda \in \Lambda} A_\lambda\)
のように書くこともある.和集合,共通部分,補集合について次の重要な定理が成立つ.

\hypertarget{properties-of-sets}{%
\paragraph{定理}\label{properties-of-sets}}
\addcontentsline{toc}{paragraph}{定理}

\(\Lambda\) を添字集合とした,集合 \(X\) の部分集合族
\(\{ A_\lambda \,:\,\lambda \in \Lambda \}\) と \(X\) の部分集合
\(B \subset X\) に対して次が成立つ: \begin{align}
&B \cup \left( \bigcup A_\lambda \right) = \bigcup ( B \cup A_\lambda ), &\quad & B \cap \left( \bigcap A_\lambda \right) = \bigcap ( B \cap A_\lambda );\\
&B \cup \left( \bigcap A_\lambda \right) = \bigcap ( B \cup A_\lambda ), &\quad & B \cap \left( \bigcup A_\lambda \right) = \bigcup ( B \cap A_\lambda );\\
&B \setminus \left( \bigcup A_\lambda \right) = \bigcap ( B \setminus A_\lambda ), &\quad & B \setminus \left( \bigcap A_\lambda \right) = \bigcup ( B \setminus A_\lambda ).
\end{align} (4)を\textbf{結合律}(\emph{associative
laws}),(5)を\textbf{分配律}(\emph{distributive
laws}),(6)を\textbf{de Morganの法則}(\emph{de Morgan's
laws})とよぶ.\(\Box\)

\hypertarget{ux8a3cux660e}{%
\paragraph{証明}\label{ux8a3cux660e}}
\addcontentsline{toc}{paragraph}{証明}

\hypertarget{ux306eux8a3cux660e}{%
\subparagraph{(4)の証明}\label{ux306eux8a3cux660e}}
\addcontentsline{toc}{subparagraph}{(4)の証明}

まず
\(B \cup \left( \bigcup A_\lambda \right) = \bigcup ( B \cup A_\lambda )\)
を示す.\(x \in B \cup \left( \bigcup A_\lambda \right)\)
とする.\(x \in B\) のときは,とくに任意の \(\lambda \in \Lambda\)
に対して \(x \in B \cup A_\lambda\)
だから,\(x \in \bigcup ( B \cup A_\lambda )\).\(x \in \bigcup A_\lambda\)
のとき,ある \(A_{\lambda'}\) について
\(x \in A_{\lambda'}\).よって,\(x \in B \cup A_{\lambda'}\)
でもあるから \(x \in \bigcup ( B \cup A_\lambda )\).これで
\(B \cup \left( \bigcup A_\lambda \right) \subset \bigcup ( B \cup A_\lambda )\)
が示せた.\(x \in \bigcup ( B \cup A_\lambda )\) とすると,ある
\(\lambda' \in \Lambda\) が存在して
\(x \in B \cup A_{\lambda'}\).\(x \in A_{\lambda'}\) ならば
\(x \in \bigcup A_\lambda\) であるから,\(x \in B\) と
\(x \in A_{\lambda'}\) のいずれの場合も
\(x \in B \cup \left( \bigcup A_\lambda \right)\) となる.よって
\(B \cup \left( \bigcup A_\lambda \right) \supset \bigcup ( B \cup A_\lambda )\)
であるから,\(B \cup \left( \bigcup A_\lambda \right) = \bigcup ( B \cup A_\lambda )\)
がいえた.

次に
\(B \cap \left( \bigcap A_\lambda \right) = \bigcap ( B \cap A_\lambda )\)
を示す.\(x \in B \cap \left( \bigcap A_\lambda \right)\)
とすると,\(x \in \bigcap A_\lambda\) であるから,すべての \(A_\lambda\)
は \(x\) を含む.\(x \in B\) と合わせて,\(x\) がすべての
\(B \cap A_\lambda\) に属することが分かる.すなわち
\(x \in \bigcap ( B \cap A_\lambda )\)
となり,\(B \cap \left( \bigcap A_\lambda \right) \subset \bigcap ( B \cap A_\lambda )\)
が成立つ.一方 \(x \in \bigcap ( B \cap A_\lambda )\) とすればすべての
\(\lambda \in \Lambda\) に対して \(x \in B \cap A_\lambda\)
だから,\(x \in B\) かつ \(x \in \bigcap A_\lambda\) である.以上より
\(B \cap \left( \bigcap A_\lambda \right) \supset \bigcap ( B \cap A_\lambda )\)
となり,\(B \cap \left( \bigcap A_\lambda \right) = \bigcap ( B \cap A_\lambda )\)
が示せた.

\hypertarget{ux306eux8a3cux660e-1}{%
\subparagraph{(5)の証明}\label{ux306eux8a3cux660e-1}}
\addcontentsline{toc}{subparagraph}{(5)の証明}

まず
\(B \cup \left( \bigcap A_\lambda \right) = \bigcap ( B \cup A_\lambda )\)
を示す.\(x \in B \cup \left( \bigcap A_\lambda \right)\)
とする.\(x \in B\) ならばすべての \(\lambda \in \Lambda\) に対して
\(x \in B \cup A_\lambda\)
であるから,\(x \in \bigcap ( B \cup A_\lambda )\).\(x \in \bigcap A_\lambda\)
のときもすべての \(\lambda \in \Lambda\) に対して
\(x \in B \cup A_\lambda\) であるから
\(x \in \bigcap ( B \cup A_\lambda )\)
となり,\(B \cup \left( \bigcap A_\lambda \right) \subset \bigcap ( B \cup A_\lambda )\)
がいえる.\(x \in \bigcap ( B \cup A_\lambda )\)
とすると,\(x \in B \cup A_\lambda\) がすべての \(\lambda\)
について成立つ.\(x \notin B\) のときは \(x \in A_\lambda\)
となるから,\(x \in \bigcap A_\lambda\).よって,\(B \cup \left( \bigcap A_\lambda \right) \supset \bigcap ( B \cup A_\lambda )\)
であるから結局
\(B \cup \left( \bigcap A_\lambda \right) = \bigcap ( B \cup A_\lambda )\)
が示せた.

次に
\(B \cap \left( \bigcup A_\lambda \right) = \bigcup ( B \cap A_\lambda )\)
を示す.\(x \in B \cap \left( \bigcup A_\lambda \right)\)
とすると,\(x\) はある \(A_{\lambda'}\) に属する.よって
\(x \in B \cap A_{\lambda'}\) であるから
\(x \in \bigcup ( B \cap A_\lambda )\),従って
\(B \cap \left( \bigcup A_\lambda \right) \subset \bigcup ( B \cap A_\lambda )\).\(x \in \bigcup ( B \cap A_\lambda )\)
ならばある \(\lambda' \in \Lambda\) について
\(x \in B \cap A_{\lambda'}\).このとき \(x \in A_{\lambda'}\)
であるから \(x \in \bigcup A_\lambda\)
となり,\(x \in B \cap \left( \bigcup A_\lambda \right)\)
がいえる.よって
\(B \cap \left( \bigcup A_\lambda \right) \supset \bigcup ( B \cap A_\lambda )\)
が従い,\(B \cap \left( \bigcup A_\lambda \right) = \bigcup ( B \cap A_\lambda )\)
を示した.

\hypertarget{ux306eux8a3cux660e-2}{%
\subparagraph{(6)の証明}\label{ux306eux8a3cux660e-2}}
\addcontentsline{toc}{subparagraph}{(6)の証明}

まず
\(B \setminus \left( \bigcup A_\lambda \right) = \bigcap ( B \setminus A_\lambda )\)
を示す.\(x \in B \setminus \left( \bigcup A_\lambda \right)\)
とすると,\(x \in B\) であって \(x\) はどの \(A_\lambda\)
にも属さない.よって任意の \(\lambda \in \Lambda\) に対して
\(x \in B \setminus A_\lambda\)
であるから,\(x \in \bigcap ( B \setminus A_\lambda )\).ここから
\(B \setminus \left( \bigcup A_\lambda \right) \subset \bigcap ( B \setminus A_\lambda )\)
が分かる.\(x \in \bigcap ( B \setminus A_\lambda )\) のとき \(x\) は
\(B\) には属するがどの \(A_\lambda\) にも属さない.すなわち
\(x \in B \setminus \left( \bigcup A_\lambda \right)\)
となり,\(B \setminus \left( \bigcup A_\lambda \right) \supset \bigcap ( B \setminus A_\lambda )\)
が分かる.以上より,\(B \setminus \left( \bigcup A_\lambda \right) = \bigcap ( B \setminus A_\lambda )\).

最後に
\(B \setminus \left( \bigcap A_\lambda \right) = \bigcup ( B \setminus A_\lambda )\)
を示す.\(x \in B \setminus \left( \bigcap A_\lambda \right)\)
に対して,\(x \in B\) であるが \(x\) の属さない \(A_{\lambda'}\)
が存在するから \(x \in \bigcup ( B \setminus A_\lambda )\).よって
\(B \setminus \left( \bigcap A_\lambda \right) \subset \bigcup ( B \setminus A_\lambda )\).\(x \in \bigcup ( B \setminus A_\lambda )\)
のときはある \(\lambda' \in \Lambda\) について
\(x \in B \setminus A_{\lambda'}\) である.とくに
\(x \notin A_{\lambda'}\) より \(x \notin \bigcap A_\lambda\)
が従うから,\(x \in B \setminus \left( \bigcap A_\lambda \right)\).よって
\(B \setminus \left( \bigcap A_\lambda \right) \supset \bigcup ( B \setminus A_\lambda )\)
がいえて,従って
\(B \setminus \left( \bigcap A_\lambda \right) = \bigcup ( B \setminus A_\lambda )\).
\(\Box\)

\hypertarget{ux76f4ux7a4dux3068ux5199ux50cf}{%
\subsection{直積と写像}\label{ux76f4ux7a4dux3068ux5199ux50cf}}

2つの集合 \(X\) と \(Y\) が与えられたとき,\(X\) と \(Y\)
の\textbf{直積}(\emph{direct product})とよばれる集合 \(X \times Y\)
が次のように定義される.まず,\(a \in X\) と \(b \in Y\)
から,順序付けられた組 \((a,b)\) を考える.この \((a,b)\) のことを2元
\(a\) と \(b\) からなる\textbf{順序対}(\emph{ordered
pair})という.ここで,「順序付けられた」とは,\(a\) と \(b\)
を並べる順序を入れ替えることができないことをいう.すなわち,\((a,b)\) と
\((b,a)\)
は一般には一致しない.そこで,\(X \times Y = \{ (a,b) \,:\,a \in X,\ b \in Y \}\)
とおき,これを \(X\) と \(Y\) の直積とする.\(X \times Y\) の2つの元
\((a,b), (a',b') \in X \times Y\) は,\(a = a'\) かつ \(b = b'\)
のとき等しいとする.したがって,\(a = b\) のとき,かつそのときに限り
\((a,b) = (b,a)\) となる.\((a,b) \in X \times Y\) に対して,\(X\) の元
\(a\) を \((a,b)\) の\textbf{第1成分}(\emph{first component}),\(Y\)
の元 \(b\) を\textbf{第2成分}(\emph{second component})とよぶ.
\(X \times Y\) の部分集合 \(R \subset X \times Y\) は,2つの元
\(a \in X\) と \(b \in Y\)
との間の\textbf{関係}(\emph{relationship})を定める.\((a,b) \in R\)
のとき \(aRb\) と書き,\(a\) と \(b\) は\textbf{関係 \(R\)
をもつ}という.\(R\) の元の第1成分全体の集合を \(R\)
の\textbf{定義域}(\emph{domain}),第2成分全体の集合を\textbf{値域}(\emph{range})とよぶ.

2つの集合 \(X\) と \(Y\) の間の関係 \(f \subset X \times Y\) が \(X\)
全体を定義域として,かつ \(afb,\ afc\) ならば \(b = c\) が成立つとき
\(f \colon X \to Y\) と書いて \(f\) を \(X\) から \(Y\)
への\textbf{写像}(\emph{map},
\emph{mapping}),\textbf{関数}(\emph{function})あるいは\textbf{対応}(\emph{correspondence})という.関数という語は,主に
\(Y = \boldsymbol R\) や \(Y = \boldsymbol C\) のときに用いる. \(afb\)
のとき \(f(a) = b\) や \(fa = b\),\(f \colon a \mapsto b\)
などと書き,\(b\) を \(a\) での \(f\)
の\textbf{値}(\emph{value})あるいは \(a\) の \(f\)
による\textbf{像}(\emph{image})という.\(X = \boldsymbol N\) のときは
\(f(a)\) を \(f_a\) と書いて \(f\) を \(Y\)
上の点列ということも多い.\(f(a)\) を``\(f\) of
\(a\)''なとという.また,\(f\) は \(a\) を \(b\)
に\textbf{写像する}(\emph{maps \(a\) to \(b\)}, \emph{assigns \(b\) to
\(a\)}),\(b\) は \(a\) に\textbf{対応する}(\emph{corresponds to
\(a\)})ともいう. \(X\) の部分集合 \(A \subset X\)
に対し,\(f(A) = \{ f(a) \,:\,a \in A \} \subset Y\) とおいて \(f\)
による \(A\) の\textbf{像}(\emph{image})という.とくに,\(f(X)\) は
\(f\) の値域となる.\(Y\) の部分集合 \(B \subset Y\)
に対して,\(f^{-1}(B) = \{ x \in X \,:\,f(x) \in B \} \subset X\) を
\(f\) による \(B\) の\textbf{逆像}(\emph{inverse
image})とよぶ.シングルトン \(\{ y \} \subset Y\)
に対してはその逆像を単に \(f^{-1}(y)\) で表す. \(f\) の値域が \(Y\)
全体であるとき,すなわち \(f(X) = Y\) であるとき,\(f\) を \(X\) から
\(Y\) への\textbf{全射}(\emph{surjection})あるいは \(X\) から \(Y\)
の\textbf{上への写像}(\emph{onto \(Y\)})という.\(X\)
に属する任意の2元 \(x, x' \in X\) に対して,\(f(x) = f(x')\)
ならばいつも \(x = x'\) であるとき,\(f\)
を\textbf{単射}(\emph{injection})という.これは,任意の \(y \in Y\)
に対して \(f^{-1}(y)\)
は空集合かシングルトンのいずれかであるとき,と言っても同じである.写像
\(f \colon X \to Y\) が全射であり同時に単射でもあるとき,\(f\) を \(X\)
から \(Y\) への\textbf{全単射}(\emph{bijection})あるいは \(X\) と
\(Y\) の間の\textbf{一対一対応}(\emph{one-to-one
correspondence})という.

\(f\) が全単射ならば,\(Y\) の任意の元 \(y \in Y\) に対してただ1つ \(X\)
の元 \(x \in X\) が定まり \(f(x) = y\) が成立つ.この \(y \in Y\) に
\(f(x) = y\) なる \(x \in X\) を対応させる \(Y\) から \(X\) への写像を
\(f\)
の\textbf{逆写像}(\emph{inverse})といい,\(f^{-1} \colon Y \to X\)
で表す.すると,\(f^{-1}(y) = x\) であるときかつそのときに限って
\(f(x) = y\) となる. \(X\) から \(Y\) への写像 \(f \colon X \to Y\) と
\(X\) の部分集合 \(A \subset X\) が与えられたときに,\(x \in A\) に
\(f(x)\) を対応させることで \(A\) から \(Y\)
への写像を定義できる.この写像を \(f\) の \(A\)
への\textbf{制限}(\emph{restriction})といい \(f \mid A\)
と書く.一般に \(f\) と \(f \mid A\) は(\(X = A\)
でない限り)定義域が異なるため,別の写像として区別する.
同じ定義域の2つの写像 \(f, g \colon X \to Y\) が等しいとは,すべての
\(x \in X\) に対して \(f(x) = g(x)\) が成立つことであり,\(f = g\)
と書く.

\hypertarget{ux4f8b-1}{%
\paragraph{例}\label{ux4f8b-1}}
\addcontentsline{toc}{paragraph}{例}

\begin{enumerate}
\def\labelenumi{\roman{enumi})}
\tightlist
\item
  空でない集合 \(X\) に対して,\(f(x) = x\) によって定まる全単射
  \(f \colon X \to X\) が存在する.この写像を \(X\)
  の\textbf{恒等写像}(\emph{identity})といい,\(\mathrm{id}_X,\ 1_X\)
  などと書く.\(X\) が明らかな場合は省略する.
\item
  空でない集合 \(X\) とその部分集合 \(A \subset X\)
  が与えられたとき,恒等写像の \(A\) への制限
  \(1_X \mid A \colon A \to X\) を \(A\) から \(X\)
  への\textbf{包含写像}(\emph{inclusion})といい,\(A \hookrightarrow X\)
  と表す.包含写像は最も簡単な単射でもあるから,\textbf{標準的単射}(\emph{canonical
  injection})ともよばれる.\(\Box\)
\end{enumerate}

2つの写像 \(f \colon X \to Y,\ g \colon Y \to Z\)
があるとき,\(x \mapsto g(f(x))\) によって \(X\) から \(Z\)
への写像が定まる.これを \(f\) と \(g\)
の\textbf{合成写像}(\emph{composite})といい,\(g \circ f\) あるいは
\(gf\) と書く.\(f\) が全単射のときは
\(f^{-1} \circ f = 1_X,\ f \circ f^{-1} = 1_Y\) となる. 写像
\(f \colon X \to Y\) が与えられたとき,容易に分かるように,\(f\)
が全単射ならば \(f^{-1}\) も全単射でさらに \((f^{-1})^{-1} = f\)
が成立つ.2つの全単射 \(f \colon X \to Y\) および \(g \colon Y \to Z\)
に対して \(gf \colon X \to Z\) が全単射になることもまた簡単に分かる.

\(\Lambda\) を添字集合とする集合族
\(\{ A_\lambda \,:\,\lambda \in \Lambda \}\) を考える.\(\Lambda\) から
\(\bigcup A_\lambda\) への写像
\(f \colon \Lambda \to \bigcup A_\lambda\) で,各
\(\lambda \in \Lambda\) に対して \(f(\lambda) \in A_\lambda\)
を満たすもの全体の集合を集合族 \(\{ A_\lambda \}\)
の\textbf{直積}(\emph{direct
product})といい,\(\prod_{\lambda \in \Lambda} A_\lambda\)
と書く.\(f \in \prod A_\lambda\) に対し,\(f(\lambda)\) の代わりに
\(f_\lambda\) と書き,\(f\) の \textbf{\(\lambda\)
成分}(\emph{\(\lambda\)-coordinate})という. \(\Lambda = \{1, 2\}\)
のとき,\(\prod A_\lambda\) と \(A_1 \times A_2\)
は同じであることが分かる.実際,\((a,b) \in A_1 \times A_2\)
であるとは,対応 \(f(1) = a,\ f(2) = b\) が \(\prod A_\lambda\)
に属していることを意味する.逆に \(f \in \prod A_\lambda\)
のとき,\((f_1, f_2)\) は順序対をなしていて
\((f_1, f_2) \in A_1 \times A_2\) である.

写像と和集合,共通部分,補集合の間に次の重要な性質がある.

\hypertarget{ux5b9aux7406}{%
\paragraph{定理}\label{ux5b9aux7406}}
\addcontentsline{toc}{paragraph}{定理}

2つの集合 \(X, Y\) と \(X\) の部分集合族
\(\{A_\lambda \subset X \,:\,\lambda \in \Lambda \}\) および \(X\) から
\(Y\) への写像 \(f \colon X \to Y\)
が与えられたとする.このとき次が成立つ:

\begin{align}
&A_1 \subset A_2 \Longrightarrow f(A_1) \subset f(A_2);\\
&f(\bigcup A_\lambda) = \bigcup f(A_\lambda);\\
&f(\bigcap A_\lambda) \subset \bigcap f(A_\lambda);\\
&f(A_1 \setminus A_2) \supset f(A_1) \setminus f(A_2).
\end{align}

さらに \(f\) が全単射のときは,\(Y\) の部分集合族
\(\{B_\lambda \subset Y \,:\,\lambda \in \Lambda \}\)
に対して次が成立つ:

\begin{align}
&B_1 \subset B_2 \Longrightarrow f^{-1}(B_1) \subset f^{-1}(B_2);\\
&f^{-1}(\bigcup B_\lambda) = \bigcup f^{-1}(B_\lambda);\\
&f^{-1}(\bigcap B_\lambda) = \bigcap f^{-1}(B_\lambda);\\
&f^{-1}(B_1 \setminus B_2) = f^{-1}(B_1) \setminus f^{-1}(B_2).
\end{align}

また,\(f\) が単射ならば(8)と(9)において等号が成立つ.\(\Box\)

\hypertarget{ux8a3cux660e-1}{%
\paragraph{証明}\label{ux8a3cux660e-1}}
\addcontentsline{toc}{paragraph}{証明}

\hypertarget{ux306eux8a3cux660e-3}{%
\subparagraph{(7)の証明}\label{ux306eux8a3cux660e-3}}
\addcontentsline{toc}{subparagraph}{(7)の証明}

\(A_1 \subset A_2\) とする.\(y \in f(A_1)\) のとき,\(x \in A_1\)
が存在して \(y = f(x)\) となる.このときまた \(x \in A_2\) でもあるから
\(y \in f(A_2)\).よって \(f(A_1) \subset f(A_2)\) となる.

\hypertarget{ux306eux8a3cux660e-4}{%
\subparagraph{(8)の証明}\label{ux306eux8a3cux660e-4}}
\addcontentsline{toc}{subparagraph}{(8)の証明}

\(y \in f(\bigcup A_\lambda)\) とすると,ある \(\lambda' \in \Lambda\)
と \(x \in A_{\lambda'}\) が存在して \(y = f(x)\) となる.よって
\(y \in f(A_{\lambda'}) \subset \bigcup f(A_\lambda)\) であるから
\(f(\bigcup A_\lambda) \subset \bigcup f(A_\lambda)\).

逆に \(y \in \bigcup f(A_\lambda)\) のときはある
\(\lambda' \in A_\lambda\) に対して \(y \in f(A_{\lambda'})\),従って
\(y = f(x)\) となる \(x \in A_{\lambda'}\) が存在する.このとき
\(x \in \bigcup A_\lambda\) であるから \(y \in f(\bigcup A_\lambda)\).

以上より \(f(\bigcup A_\lambda) = \bigcup f(A_\lambda)\).

\hypertarget{ux306eux8a3cux660e-5}{%
\subparagraph{(9)の証明}\label{ux306eux8a3cux660e-5}}
\addcontentsline{toc}{subparagraph}{(9)の証明}

\(y \in f(\bigcap A_\lambda)\) とすると \(y = f(x)\) となる
\(x \in \bigcap A_\lambda\) が存在する.すると各 \(\lambda\) に対して
\(x \in A_\lambda\) であるから \(y \in f(A_\lambda)\) が成立ち,よって
\(y \in \bigcap f(A_\lambda)\) となる.これで
\(f(\bigcap A_\lambda) \subset \bigcap f(A_\lambda)\) を示せた.

\(f\) が単射で \(y \in \bigcap f(A_\lambda)\) ならば,任意の \(\lambda\)
に対して \(y \in f(A_\lambda)\) であり,\(y = f(x_\lambda)\) となる
\(x_\lambda \in A_\lambda\) が存在する.\(f\) の単射性より \(x_\lambda\)
はすべて等しく,その値を \(x\) とおく.このとき任意の \(\lambda\)
に対して \(x \in A_\lambda\) だから \(x \in \bigcap A_\lambda\).これと
\(y = f(x)\) より \(y \in f(\bigcap A_\lambda)\)
となるから,\(f(\bigcap A_\lambda) \supset \bigcap f(A_\lambda)\).

\hypertarget{ux306eux8a3cux660e-6}{%
\subparagraph{(10)の証明}\label{ux306eux8a3cux660e-6}}
\addcontentsline{toc}{subparagraph}{(10)の証明}

\(y \in f(A_1) \setminus f(A_2)\) とすると,\(y \in f(A_1)\) より
\(y = f(x)\) なる \(x \in A_1\) が存在する.一方 \(y \notin f(A_2)\)
であるから \(x \in A_2\) ではあり得ない.よって
\(x \in A_1 \setminus A_2\)
であり,\(y \in f(A_1 \setminus A_2)\).すなわち
\(f(A_1 \setminus A_2) \supset f(A_1) \setminus f(A_2)\) が分かる.

\(f\) が単射のときは,\(y \in f(A_1 \setminus A_2)\) ならば \(y = f(x)\)
となる \(x \in A_1 \setminus A_2\) が存在するから \(y \in f(A_1)\)
は分かる.ここで \(y \in f(A_2)\) とするとある \(x' \in A_2\) が存在して
\(y = f(x')\) であるが,\(f\) の単射性より
\(x = x' \in A_1 \cap A_2\).これは \(x \in A_1 \setminus A_2\)
に反する.よって \(y \notin f(A_2)\) であるから
\(y \in f(A_1) \setminus f(A_2)\),従って
\(f(A_1 \setminus A_2) \subset f(A_1) \setminus f(A_2)\).

\hypertarget{ux306eux8a3cux660e-7}{%
\subparagraph{(11), (12), (13), (14)の証明}\label{ux306eux8a3cux660e-7}}
\addcontentsline{toc}{subparagraph}{(11), (12), (13), (14)の証明}

\(f\) が全単射であるから \(f^{-1}\) も全単射.よって(7), (8), (9),
(10)において \(f \colon X \to Y\) を \(f^{-1} \colon Y \to X\)
に,\(\{ A_\lambda \}\) を \(\{ B_\lambda \}\)
に置き換えれば直ちに従う.\(\Box\)

写像の単射性と全射性に関する次の事実がよく用いられる.

\hypertarget{equivalent-statement-on-injectivity-and-surjectivity}{%
\paragraph{定理}\label{equivalent-statement-on-injectivity-and-surjectivity}}
\addcontentsline{toc}{paragraph}{定理}

写像 \(f \colon X \to Y\) に対して次が成立つ:

\begin{enumerate}
\def\labelenumi{\roman{enumi})}
\tightlist
\item
  \(f\) が単射であることの必要十分条件は,\(gf = 1_X\) なる写像
  \(g \colon Y \to X\) が存在することである;
\item
  \(f\) が全射であることの必要十分条件は,\(fh = 1_Y\) なる写像
  \(h \colon Y \to X\) が存在することである.\(\Box\)
\end{enumerate}

\hypertarget{ux8a3cux660e-2}{%
\paragraph{証明}\label{ux8a3cux660e-2}}
\addcontentsline{toc}{paragraph}{証明}

\hypertarget{i.ux306eux8a3cux660e}{%
\subparagraph{i.の証明}\label{i.ux306eux8a3cux660e}}
\addcontentsline{toc}{subparagraph}{i.の証明}

必要性を示す.\(f\) が単射ならば,任意の2元 \(x, x' \in X\) に対して
\(f(x) = f(x')\) ならば \(x = x'\) である.よって,\(X\) の元
\(x_0 \in X\) を適当に固定すれば,写像 \(g \colon Y \to X\) が
\begin{align*}
g(y) = \left\{ \begin{array}{ll} x & (\text{if}\ y = f(x)), \\ x_0 & (\text{if}\ y \notin f(X) \end{array}  \right.
\end{align*}
によって定義できる(これが写像であることは確かめなければならない.すなわち,\(f(x) = f(x')\)
ならば \(gf(x) = gf(x')\) を確かめる必要があるが,これは \(f\)
の単射性より明らか).そうすると,任意の \(x \in X\) に対して
\(gf(x) = x\) だから,\(gf = 1_X\).

次に十分性を示す.写像 \(g \colon Y \to X\) が存在して \(gf = 1_X\)
ならば,任意の \(x, x' \in X\) に対して \(x = gf(x)\) および
\(x' = gf(x')\) が成立つ.ここで \(f(x) = f(x')\)
とすれば,\(gf(x) = gf(x')\) より \(x = x'\) が分かる.これは \(f\)
が単射であることを意味する.

\hypertarget{ii.ux306eux8a3cux660e}{%
\subparagraph{ii.の証明}\label{ii.ux306eux8a3cux660e}}
\addcontentsline{toc}{subparagraph}{ii.の証明}

\(f\) を全射とすると,任意の \(y \in Y\) に対して \(f(x) = y\) なる
\(x \in X\) が少なくとも1つ存在する.そのうちの1つを適当に選び \(x_y\)
とする.そこで写像 \(h \colon Y \to X\) を \(h(y) = x_y\)
と定義すれば,\(fh(y) = f(x_y) = y\) であるから \(fh = 1_Y\) となる.

逆に \(fh = 1_Y\) となる写像 \(h \colon Y \to X\)
が存在すると仮定する.このとき,任意の \(y \in Y\) に対して
\(y = f(h(y))\) であるから,\(f\) は全射.\(\Box\)

写像の単射性と全射性を圏論的に言い換えると以下のようになる.

\hypertarget{ux5b9aux7406-1}{%
\paragraph{定理}\label{ux5b9aux7406-1}}
\addcontentsline{toc}{paragraph}{定理}

写像 \(f \colon X \to Y\) に対して次が成立つ:

\begin{enumerate}
\def\labelenumi{\roman{enumi})}
\tightlist
\item
  \(f\) が単射であることの必要十分条件は,\(fg_1 = fg_2\) なる任意の写像
  \(g_1, g_2 \colon Z \to X\) に対して \(g_1 = g_2\) となることである;
\item
  \(f\) が全射であることの必要十分条件は,\(h_1f = h_2f\) なる任意の写像
  \(h_1, h_2 \colon Y \to Z\) に対して \(h_1 = h_2\)
  となることである.\(\Box\)
\end{enumerate}

\hypertarget{ux8a3cux660e-3}{%
\paragraph{証明}\label{ux8a3cux660e-3}}
\addcontentsline{toc}{paragraph}{証明}

\hypertarget{i.ux306eux8a3cux660e-1}{%
\subparagraph{i.の証明}\label{i.ux306eux8a3cux660e-1}}
\addcontentsline{toc}{subparagraph}{i.の証明}

\(f\) を単射として,2つの写像 \(g_1, g_2 \colon Z \to X\) について
\(fg_1 = fg_2\) となったとする.このとき \(g_1 = g_2\)
を示したい.任意の \(z \in Z\) に対して \(f(g_1(z)) = f(g_2(z))\)
であるから,\(f\) の単射性より \(g_1(z) = g_2(z)\) となる.よって
\(g_1 = g_2\).

逆に任意の写像 \(g_1, g_2 \colon Z \to X\) に対して,\(fg_1 = fg_2\)
ならば \(g_1 = g_2\) が成立つとする.このとき特に \(Z = \{ 0 \}\)
として,\(g_1(0) = x,\ g_2(0) = y\) とする.もし \(f(x) = f(y)\)
であれば,\(fg_1(0) = fg_2(0)\) であるから \(g_1 = g_2\)
が言える.従って \(x = g_1(0) = g_2(0) = y\) となり,\(f\) は単射.

\hypertarget{ii.ux306eux8a3cux660e-1}{%
\subparagraph{ii.の証明}\label{ii.ux306eux8a3cux660e-1}}
\addcontentsline{toc}{subparagraph}{ii.の証明}

まず \(f\) が全射ならば,\(h_1f = h_2f\) なる写像
\(h_1, h_2 \colon Y \to Z\) に対して \(h_1 = h_2\) となる.実際,任意の
\(y \in Y\) に対して \(x \in X\) が存在して \(y = f(x)\)
とでき,このとき \(h_1(y) = h_1f(x) = h_2f(x) = h_2(y)\) より
\(h_1 = h_2\).

次に任意の写像 \(h_1, h_2 \colon Y \to Z\) に対して,\(h_1f = h_2f\)
ならば \(h_1 = h_2\) となるとする.\(f\) が全射でないとすると
\(y_0 \in Y \setminus f(X)\) が存在する.そこで \(Z = \{ 0, 1 \}\)
として \begin{align*}
h_1(y) &= 0,\\
h_2(y) &= \left\{ \begin{array}{ll} 0 & (\text{if}\ y \neq y_0),\\ 1 & (\text{if}\ y = y_0) \end{array} \right.
\end{align*} と定義すれば,\(h_1f = h_2f\) であるが \(h_1 \neq h_2\)
となり矛盾.よって \(f\) は全射である.\(\Box\)

\hypertarget{equivalence-of-existence-of-injection-and-surjection}{%
\paragraph{系}\label{equivalence-of-existence-of-injection-and-surjection}}
\addcontentsline{toc}{paragraph}{系}

2つの集合 \(X\) と \(Y\) に対して,単射 \(\colon X \to Y\)
が存在するための必要十分条件は全射 \(\colon Y \to X\)
が存在することである.\(\Box\)

\hypertarget{ux8a3cux660e-4}{%
\paragraph{証明}\label{ux8a3cux660e-4}}

\protect\hyperlink{equivalent-statement-on-injectivity-and-surjectivity}{定理}より,単射
\(f \colon X \to Y\) が存在すれば写像 \(g \colon Y \to X\) が存在して
\(gf = 1_X\) となり,またこの \(g\) は全射である.

逆に全射 \(g \colon Y \to X\) が存在すれば同じ定理より \(gf = 1_X\)
なる写像 \(f \colon X \to Y\) が存在し,\(f\) の単射性も分かる.\(\Box\)

\hypertarget{bernstein-theorem}{%
\paragraph{Bernsteinの定理}\label{bernstein-theorem}}
\addcontentsline{toc}{paragraph}{Bernsteinの定理}

2つの集合 \(X, Y\) に対して,以下の条件はすべて同値である:

\begin{enumerate}
\def\labelenumi{\roman{enumi})}
\tightlist
\item
  全単射 \(\colon X \to Y\) が存在する;
\item
  単射 \(\colon X \to Y\) および単射 \(\colon Y \to X\) が存在する;
\item
  全射 \(\colon X \to Y\) および全射 \(\colon Y \to X\) が存在する;
\item
  単射 \(\colon X \to Y\) および全射 \(\colon X \to Y\) が存在する;
\item
  \(X\) から \(Y\) の部分集合への全単射が存在し,かつ \(Y\) から \(X\)
  の部分集合への全単射が存在する.\(\Box\)
\end{enumerate}

\hypertarget{ux8a3cux660e-5}{%
\paragraph{証明}\label{ux8a3cux660e-5}}
\addcontentsline{toc}{paragraph}{証明}

ii.\(\;\Leftrightarrow\) iii. \(\Leftrightarrow\) iv.
\(\Leftrightarrow\) v.
は\protect\hyperlink{equivalence-of-existence-of-injection-and-surjection}{系}からすぐに従う.

i.\(\;\Leftrightarrow\) ii. を示す. 全単射 \(f \colon X \to Y\)
が存在すれば \(f\) と \(f^{-1}\) はともに全単射だから,i.
\(\Rightarrow\) ii. はよい.2つの単射
\(f \colon X \to Y,\ g \colon Y \to X\) が存在するときに全単射
\(h \colon X \to Y\) を構成できれば ii. \(\Rightarrow\) i. も示される.

\(f\) が全射であれば \(f\) が \(X\) から \(Y\)
への全単射になっている.以下,\(f\)
は全射でないとする.\(Y_0 = Y \setminus f(X)\) として, \begin{align*}
X_n = g(Y_{n - 1}),\quad Y_n = f(X_n) \quad (n = 1, 2, 3, \dotsc)
\end{align*} とおく.もちろん \(X_n, Y_n \neq \emptyset\)
である.さらに, \begin{align*}
&X_* = \bigcup_{n = 1}^\infty X_n, &Y_* = \bigcup_{n = 0}^\infty Y_n,\\
&X^* = X \setminus X_*, &Y^* = Y \setminus Y_*
\end{align*} とおく.このとき,\(f(X^*) = Y^*,\ g(Y_*) = X_*\)
が成立つ.実際,\(f\) の単射性より \begin{align*}
f(X^*) &= f(X \setminus X_*) = f(X) \setminus f(X_*) = (Y \setminus Y_0) \setminus f(X_*) = Y \setminus (Y_0 \cup f(X_*)),\\
Y_0 \cup f(X_*) &= Y_0 \cup f( \bigcup_{n = 1}^\infty X_n ) = Y_0 \cup \left( \bigcup_{n = 1}^\infty f(X_n) \right) = \bigcup_{n = 0}^\infty Y_n = Y_*
\end{align*} であるから,\(f(X^*) = Y \setminus Y_* = Y^*\)
となる.また, \begin{align*}
g(Y_*) = g( \bigcup_{n = 0}^\infty Y_n) = \bigcup_{n = 0}^\infty g(Y_n) = \bigcup_{n = 0}^\infty X_{n + 1} = X_*.
\end{align*} よって,制限 \(f \mid X^* \colon X^* \to Y^*\) と
\(g \mid Y_* \colon Y_* \to X_*\) は全単射であり,さらに全単射
\((g \mid Y_*)^{-1} \colon X_* \to Y_*\) が存在する. そこで,
\begin{align*}
h(x) = \left\{ \begin{array}{ll} (f \mid X^*)(x) & (\text{if}\ x \in X^*), \\ (g \mid Y_*)^{-1}(x) & (\text{if}\ x \in X_*) \end{array} \right.
\end{align*} とおけば \(h \colon X \to Y\)
は全単射であることが分かる.これで ii. \(\Rightarrow\) i.
が示された.\(\Box\)

集合 \(X\) と \(Y\) に対して,\(X\) から \(Y\) への写像全体の集合を
\(Y^X\) と書く.とくに \(Y = \{ 0, 1 \}\) のときは
\(Y^X = 2^X = \mathfrak P(X)\) などと書き,\(X\)
の\textbf{冪集合}(\emph{power set})という.冪集合 \(2^X\) は,\(X\)
の部分集合全体の集合と同一視される.実際,\(X\) の部分集合 \(A\)
に対して \(A\) の\textbf{特性関数}(\emph{characteristic
function})あるいは\textbf{指示関数}(\emph{indicator
function})とよばれる写像 \(\chi_A \colon X \to \{ 0, 1 \}\)
が一対一に対応する.\(A\) の特性関数は,\(x \in A\) のとき
\(\chi_A(x) = 1\),\(x \notin A\) のとき \(\chi_A(x) = 0\)
と定義される.逆に写像 \(\chi \colon X \to \{ 0, 1 \}\) が与えられたとき
\(A = \{ x \in X \,:\,\chi(x) = 1 \}\) とおくことで,\(\chi\)
を特性関数とする部分集合 \(A\) が定まる.

\hypertarget{ux540cux5024ux95a2ux4fc2}{%
\subsection{同値関係}\label{ux540cux5024ux95a2ux4fc2}}

集合 \(X\) 上の\textbf{2項関係}(\emph{binary relation})とは,関係
\(\sim \subset X \times X\)
のことである.数学では多くの2項関係について考えるが,それらに共通する大事な性質がある.

\hypertarget{ux5b9aux7fa9}{%
\paragraph{定義}\label{ux5b9aux7fa9}}
\addcontentsline{toc}{paragraph}{定義}

\(\sim\) を集合 \(X\) 上の2項関係とする.このとき次の性質を考える:

\begin{enumerate}
\def\labelenumi{\roman{enumi})}
\tightlist
\item
  任意の \(x \in X\) に対して \(x \sim x\)
  であることを\textbf{反射律}(\emph{reflexivity})という.反射律を満たす2項関係は\textbf{反射的}(\emph{reflexive})であるという;
\item
  任意の \(x, y \in X\) に対して,\(x \sim y\) ならば \(y \sim x\)
  でもあることを\textbf{対称律}(\emph{symmetry})という.対称律を満たす2項関係は\textbf{対称}(\emph{symmetric})であるという;
\item
  任意の \(x, y \in X\) に対して,\(x \sim y\) かつ \(y \sim x\) ならば
  \(x = y\)
  となることを\textbf{反対称律}(\emph{antisymmetry})という.反対称律を満たす2項関係は\textbf{反対称}(\emph{antisymmetric})であるという;
\item
  任意の \(x, y, z \in X\) に対して,\(x \sim y\) かつ \(y \sim z\)
  のときいつも \(x \sim z\)
  であることを\textbf{推移律}(\emph{transitivity})という.推移律を満たす2項関係は\textbf{推移的}(\emph{transitive})であるという.
\end{enumerate}

反射的で推移的な対称関係を\textbf{同値関係}(\emph{equivalence
relation})という.反射的で推移的な反対称関係を\textbf{半順序関係}(\emph{partial
order})といい,半順序関係が1つ定まった集合を\textbf{半順序集合}(\emph{partially
ordered set},
\emph{poset})という.誤解のない場合は,半順序関係,半順序集合をそれぞれ単に\textbf{順序関係}(\emph{order}),\textbf{順序集合}(\emph{ordered
set})という.半順序はよく \(\leqslant\) で表される. \((X, \leqslant)\)
を半順序集合とするとき,2つの元 \(x, y \in X\) であって
\(x \leqslant y\) でも \(y \leqslant x\)
でもないようなものが存在することもある.\(X\) の2元 \(x, y \in X\)
は,\(x \leqslant y\) または \(y \leqslant x\)
であるとき\textbf{比較可能}(\emph{comparable})といい,比較可能でないとき\textbf{比較不可能}(\emph{incomparable})という.\(X\)
の任意の2元が比較可能であるとき,半順序 \(\leqslant\)
を\textbf{全順序}(\emph{total order})といい,\(X\)
を\textbf{全順序集合}(\emph{totally ordered set})という.\(\Box\)

\hypertarget{ux4f8b-2}{%
\paragraph{例}\label{ux4f8b-2}}
\addcontentsline{toc}{paragraph}{例}

\begin{enumerate}
\def\labelenumi{\roman{enumi})}
\tightlist
\item
  実数上の相等関係 \(=\) は同値関係で,通常の順序 \(\leqslant\)
  は全順序である.\(\geqslant\) も全順序である.
\item
  \(n \in \boldsymbol Z_{> 0}\) を1つ固定する.\(\boldsymbol Z\)
  上に,\(n \mid (x - y)\) によって定まる2項関係
  \(\equiv \ (\mathrm{mod}\ n)\) は同値関係である(\(a \mid b\) は \(a\)
  が \(b\) を割り切ることを表す.\(a\) divides \(b\).).また
  \(x \mid y\) は \(\boldsymbol Z_{> 0}\)
  上の半順序であるが全順序ではない.
\item
  \(X\) を集合として,\(X\) の部分集合全体の集合を
  \(2^X = \{ A \,:\,A \subset X \}\) と書く.\(X\)
  の部分集合間の包含関係 \(\subset\) は \(2^X\)
  上の半順序であるが全順序ではない.\(X\) の部分集合 \(A, B \subset X\)
  に対して \(A \sim B\) を \(A\) から \(B\)
  への全単射が存在することとすると,\(\sim\) は \(2^X\)
  上の同値関係である.
\item
  三角形全体の集合を \(T\) とする.三角形の合同関係 \(\equiv\)
  および相似関係 \(\approx\) はともに \(T\) 上の同値関係となる.
\item
  \(\boldsymbol R^{n\times} = \boldsymbol R^n \setminus \{0\}\)
  とおく.\(x = (x_1, \dotsc, x_n), y = (y_1, \dotsc, y_n) \in \boldsymbol R^{n \times}\)
  に対して,\(x \sim y\) をある正数 \(c, c'\) が存在して \(cx = c'y\)
  とできることとする.すなわち,\(x \sim y \Leftrightarrow \exists c, c' \in \boldsymbol R_{> 0},\ (c x_1, \dotsc, c x_n) = (c' y_1, \dotsc, c' y_n)\).このとき,\(\sim\)
  は \(\boldsymbol R^{n \times}\) 上の同値関係を定める.\(\Box\)
\end{enumerate}

大雑把に言えば,同値関係は集合の元が「同じと見なせる」ことを定める.厳密には異なる元でも本質的には変わらないなら区別する必要はない,という考え方を定式化したのが同値関係であり同値類である.

今集合 \(X\) 上に同値関係 \(\sim\) が1つ定まっているとする.\(X\) の元
\(x \in X\) に対して,「\(x\) と本質的に同じ」元の集合を \(\overline x\)
あるいは \([x]\) と書き \(x\) の\textbf{同値類}(\emph{equivalence
class})という.すなわち, \[
\overline x = [x] = \{ y \in X \,:\,x \sim y \}
\] を \(x\) の同値類とよぶ.同値類全体の集合
\(\{ \overline x \,:\,x \in X \}\) を \(X\) の \(\sim\)
による\textbf{商集合}(\emph{quotient set})といい,\({X}/\!\,{\sim}\)
と書く.定義より, \[
\overline x = \overline y \Longleftrightarrow x \sim y
\] が分かる.\(X\) から \({X}/\!\,{\sim}\) への写像
\(x \mapsto \overline x\)
は全射である.これを\textbf{標準的全射}(\emph{canonical
surjection})あるいは\textbf{標準的射影}(\emph{canonical
projection})とよぶ.各同値類 \([x]\) から1つ元を選ぶ写像
\(s \colon {X}/\!\,{\sim }\to X\)
を\textbf{切断}(\emph{section})とよぶ(正確には,\(s\)
は切断の1つである).すなわち,\(s \colon {X}/\!\,{\sim }\to X\)
が切断であるとは,任意の \(c \in {X}/\!\,{\sim}\) に対して
\(\overline{s(c)} = c\) となることをいう.\(s(c)\) を \(c\)
の\textbf{代表元}(\emph{representative})という.

\hypertarget{ux4f8b-3}{%
\paragraph{例}\label{ux4f8b-3}}
\addcontentsline{toc}{paragraph}{例}

\begin{enumerate}
\def\labelenumi{\roman{enumi})}
\tightlist
\item
  \(n \in \boldsymbol Z_{> 0}\) とする.\(\boldsymbol Z\) の
  \(\equiv\ (\mathrm{mod}\ n)\) による商集合を
  \(\boldsymbol Z/ n \boldsymbol Z\)
  と書く.\(\boldsymbol Z / n \boldsymbol Z = \{ \overline 0, \overline 1, \dotsc, \overline{n - 1} \}\)
  である.\(\overline k\) は \(k + (n)\)
  とも書かれる.これは環論における記法である.また \(\overline k\)
  を単に \(k\) と書くこともある.
\item
  先の例v.における \(\boldsymbol R^{n \times}\) 上の同値関係 \(\sim\)
  については,\(\overline x = \{ k x \,:\,k \in \boldsymbol R_{> 0} \}\)
  となる.\(\overline x\) の代表元としては,\(\| x \| = 1\)
  となるように選ぶ.量子力学では実際にこの場合を考えている.\(\Box\)
\end{enumerate}

\hypertarget{ux5b9aux7fa9-1}{%
\paragraph{定義}\label{ux5b9aux7fa9-1}}
\addcontentsline{toc}{paragraph}{定義}

集合 \(X, Y\) と写像 \(f \colon X \to Y\) が与えられたとする. \[
x \sim_f y \Longleftrightarrow f(x) = f(y) \quad (x, y \in X)
\] によって \(f\) の\textbf{同値核}(\emph{equivalence
kernel})とよばれる \(X\) 上の同値関係 \(\sim_f\) を定義する.このとき
\(\overline f(\overline x) = f(x)\) によって写像
\(\overline f \colon {X}/\!\,{\sim}_f \to Y\) が定まる.\(\overline f\)
は全単射であり,\(f\) から\textbf{誘導された}(\emph{induced from
\(f\)})全単射という.\(\Box\)

\hypertarget{ux6ce8}{%
\paragraph{注}\label{ux6ce8}}
\addcontentsline{toc}{paragraph}{注}

商集合を圏論的に考える.2つの全射
\(f \colon X \to Y,\ g \colon X \to Z\) に対して, \[
f \leqslant g \Longleftrightarrow \exists w \colon Z \to Y,\ f = wg
\] として,\(f \sim g\) を \(f \leqslant g\) かつ \(g \leqslant f\)
とすれば \(\sim\) は
\(\{ (Y, f \colon X \to Y) \,:\,Y \ \text{is a set}, \ f \ \text{is surjective} \}\)
上の同値関係になる.そこで同値類 \(\overline{(Y, f)}\)
を考えると,これは \(f \colon X \to Y\)
の同値核による商集合となる.厳密にいえば,(全射とは限らない)写像
\(f \colon X \to Y\) に対して,\(({X}/\!\,{\sim}_f, \overline f)\)
が代表元となる.\(\Box\)

\end{document}
