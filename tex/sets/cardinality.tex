\PassOptionsToPackage{unicode=true}{hyperref} % options for packages loaded elsewhere
\PassOptionsToPackage{hyphens}{url}
%
\documentclass[b5paper,pandoc]{bxjsarticle}
\usepackage{lmodern}
\usepackage{amssymb,amsmath}
\usepackage{ifxetex,ifluatex}
\usepackage{fixltx2e} % provides \textsubscript
\ifnum 0\ifxetex 1\fi\ifluatex 1\fi=0 % if pdftex
  \usepackage[T1]{fontenc}
  \usepackage[utf8]{inputenc}
  \usepackage{textcomp} % provides euro and other symbols
\else % if luatex or xelatex
  \usepackage{unicode-math}
  \defaultfontfeatures{Ligatures=TeX,Scale=MatchLowercase}
\fi
% use upquote if available, for straight quotes in verbatim environments
\IfFileExists{upquote.sty}{\usepackage{upquote}}{}
% use microtype if available
\IfFileExists{microtype.sty}{%
\usepackage[]{microtype}
\UseMicrotypeSet[protrusion]{basicmath} % disable protrusion for tt fonts
}{}
\IfFileExists{parskip.sty}{%
\usepackage{parskip}
}{% else
\setlength{\parindent}{0pt}
\setlength{\parskip}{6pt plus 2pt minus 1pt}
}
\usepackage{hyperref}
\hypersetup{
            pdftitle={集合論},
            pdfauthor={M. Nakata},
            pdfborder={0 0 0},
            breaklinks=true}
\urlstyle{same}  % don't use monospace font for urls
\setlength{\emergencystretch}{3em}  % prevent overfull lines
\providecommand{\tightlist}{%
  \setlength{\itemsep}{0pt}\setlength{\parskip}{0pt}}
\setcounter{secnumdepth}{5}
% Redefines (sub)paragraphs to behave more like sections
\ifx\paragraph\undefined\else
\let\oldparagraph\paragraph
\renewcommand{\paragraph}[1]{\oldparagraph{#1}\mbox{}}
\fi
\ifx\subparagraph\undefined\else
\let\oldsubparagraph\subparagraph
\renewcommand{\subparagraph}[1]{\oldsubparagraph{#1}\mbox{}}
\fi

% set default figure placement to htbp
\makeatletter
\def\fps@figure{htbp}
\makeatother


\title{集合論}
\author{M. Nakata}
\date{}

\def\theparagraph{}
\def\thesubparagraph{}
\def\lt{<}
\def\gt{>}
\begin{document}
\maketitle

{
\setcounter{tocdepth}{3}
\tableofcontents
}
\hypertarget{ux6fc3ux5ea6}{%
\section{濃度}\label{ux6fc3ux5ea6}}

集合論においては,集合の元が具体的に何であるかは重要ではない.したがって,\(\{ 0, 1, 2 \}\)
と \(\{ 0, -1, -2 \}\)
に違いはない.もちろん,集合としては異なるけれどこれらを区別することは本質ではない.このように,同じ個数の集合同士を区別する必要はない.集合の個数は濃度とよばれる.同じ濃度の集合は同じものとみなすのである.

\hypertarget{ux5b9aux7fa9}{%
\paragraph{定義}\label{ux5b9aux7fa9}}
\addcontentsline{toc}{paragraph}{定義}

2つの集合 \(X\) と \(Y\) は,\(X\) から \(Y\)
への全単射が存在するとき\textbf{対等}(\emph{equivalent})であるといい,\(X \cong Y\)
と表す.この \(\cong\)
は集合間の同値関係になる(ただし,「集合全体の集合」は存在しない.すべての集合を集めたものは\textbf{類}(\emph{class})とよばれる.類においても集合と同じように同値関係を考えればよい).集合
\(X\) の同値類を \(\mathrm{card}(X)\) と書き,\(X\)
の\textbf{濃度}(\emph{cardinality}, \emph{cardinal})という. \(n\)
個の元からなる有限集合 \(\{1, 2, \dotsc, n \}\) の濃度を \(n\)
で表す.さらに,\(\aleph_0\) で \(\boldsymbol N\)
の濃度を,\(\mathfrak c\) で \(\boldsymbol R\)
の濃度を表し,\(\mathfrak c\) を\textbf{連続体濃度}(\emph{cardinality
of the continuum})という.濃度が \(\aleph_0\)
であるような集合は\textbf{可算集合}(\emph{countable
set})とよばれる.集合が有限集合か可算集合のいずれかであることを\textbf{高々可算}(\emph{at
most
countable})であるという.\(\mathrm{card}(X) = \mathfrak m,\ \mathrm{card}(Y) = \mathfrak n\)
であるような集合 \(X,\ Y\) に対して,\(Y^X\) の濃度を
\(\mathfrak n^{\mathfrak m}\) で表す.
\(\{ X_\lambda \,:\,\lambda \in \Lambda \}\)
を互いに素な集合族とするとき,濃度の和と積をそれぞれ \begin{align*}
\sum_{\lambda \in \Lambda} \mathrm{card}(X_\lambda) &= \mathrm{card}(\coprod_{\lambda \in \Lambda} X_\lambda),\\
\prod_{\lambda \in \Lambda} \mathrm{card}(X_\lambda) &= \mathrm{card}(\prod_{\lambda \in \Lambda} X_\lambda)
\end{align*} と定義する.\(\Box\)

以下では,\protect\hyperlink{equivalence-of-existence-of-injection-and-surjection}{系}と\protect\hyperlink{bernstein-theorem}{Bernsteinの定理}は一々断らない.

\hypertarget{ux88dcux984c}{%
\paragraph{補題}\label{ux88dcux984c}}
\addcontentsline{toc}{paragraph}{補題}

濃度の和,積および冪乗はwell-definedである.すなわち,\(\{ X_\lambda \,:\,\lambda \in \Lambda \}, \{ Y_\lambda \,:\,\lambda \in \Lambda \}\)
をともに互いに素な集合族であって,各 \(\lambda \in \Lambda\) に対して
\(\mathrm{card}(X_\lambda) = \mathrm{card}(Y_\lambda)\)
なるものとすると, \begin{align*}
\mathrm{card}(\coprod X_\lambda) &= \mathrm{card}(\coprod Y_\lambda),\\
\mathrm{card}(\prod X_\lambda) &= \mathrm{card}(\prod Y_\lambda),\\
\mathrm{card}(X_1^{X_2}) &= \mathrm{card}(Y_1^{Y_2}).\ \Box
\end{align*}

\hypertarget{ux8a3cux660e}{%
\paragraph{証明}\label{ux8a3cux660e}}
\addcontentsline{toc}{paragraph}{証明}

仮定より,全単射の族
\(\{ f_\lambda \colon X_\lambda \to Y_\lambda \,:\,\lambda \in \Lambda \}\)
が存在する.このとき \begin{align*}
\coprod f_\lambda (x) &= f_{\mu}(x) \quad (\text{if}\ x \in X_\mu),\\
\left(\prod f_\lambda (x)\right)_\mu &= f_\mu (x_\mu)
\end{align*} によって写像
\(\coprod f_\lambda \colon \coprod X_\lambda \to \coprod Y_\lambda,\ \prod f_\lambda \colon \prod X_\lambda \to \prod Y_\lambda\)
を定義すれば,これらは全単射である.まず \(\coprod f_\lambda\)
が全単射であることを示す.任意の \(y \in Y_\mu\) に対して
\(\coprod f_\lambda(f_\mu^{-1}(y)) = y\)
だから全射である.\(\coprod f_\lambda (x) = \coprod f_\lambda (x')\)
とすれば \(x, x' \in X_\mu\) であるような \(\mu\)
が存在して,\(f_\mu (x) = f_\mu (x')\).よって \(f_\mu\) の単射性より
\(x = x'\).次に \(\prod f_\lambda\)
が全単射であることを示す.\(\prod f_\lambda (x) = \prod f_\lambda (x')\)
とすれば任意の \(\mu\) に対して \(f_\mu(x_\mu) = f_\mu (x'_\mu)\)
であるから,\(f_\mu\) が単射であることから \(x_\mu = x'_\mu\).\(\mu\)
は任意だから \(x = x'\) が従う.一方,任意の \(y \in \prod Y_\lambda\)
に対して,\(x_\lambda = f_\lambda^{-1}(y_\lambda)\) によって
\(x \in \prod X_\lambda\) を定義すれば
\(\prod f_\lambda (x) = y\).よって \(\prod f_\lambda\)
は単射でもある.以上より,\(\coprod f_\lambda\) と \(\prod f_\lambda\)
は全単射であるから, \begin{align*}
\mathrm{card}(\coprod X_\lambda) &= \mathrm{card}(\coprod Y_\lambda),\\
\mathrm{card}(\prod X_\lambda) &= \mathrm{card}(\prod Y_\lambda).
\end{align*} 最後に,\(f \colon X_2 \to X_1\) に
\(\hat f = f_1ff_2^{-1}\) を対応させれば,これが \(X_1^{X_2}\) から
\(Y_1^{Y_2}\) への全単射となる.実際,\(\hat f = \hat g\) ならば
\(f_1ff_2^{-1} = f_1gf_2^{-1}\) であるから,\(f = g\)
となる.また,\(h \colon Y_2 \to Y_1\) に対して \(f = f_1^{-1}hf_2\)
とおけば \(\hat f = h\) となる.よって写像 \(f \mapsto \hat f\)
は全単射であり, \[
\mathrm{card}(X_1^{X_2}) = \mathrm{card}(Y_1^{Y_2})
\] が言える.\(\Box\)

\hypertarget{ux88dcux984c-1}{%
\paragraph{補題}\label{ux88dcux984c-1}}
\addcontentsline{toc}{paragraph}{補題}

集合 \(X\) と \(Y\)
が与えられたとし,\(\mathfrak m = \mathrm{card}(X),\ \mathfrak n = \mathrm{card}(Y)\)
とおく.単射 \(\colon X \to Y\) が存在するとき
\(\mathfrak m \leqslant\mathfrak n\)
とすると,これは濃度の順序関係を定める.また,\(\mathfrak m \leqslant\mathfrak n\)
であるための必要十分条件は全射 \(\colon Y \to X\)
が存在することである.\(\Box\)

\hypertarget{ux8a3cux660e-1}{%
\paragraph{証明}\label{ux8a3cux660e-1}}
\addcontentsline{toc}{paragraph}{証明}

\(\mathfrak m \leqslant\mathfrak n\) の必要十分条件が全射
\(\colon Y \to X\) の存在であることはよい. \(\leqslant\)
が順序関係であることを確かめる.まず,全単射 \(1_X \colon X \to X\)
が存在するから \(\mathfrak m \leqslant\mathfrak m\)
である.\(\mathfrak m \leqslant\mathfrak n\) かつ
\(\mathfrak n \leqslant\mathfrak m\) のとき単射 \(\colon X \to Y\)
および \(\colon Y \to X\) が存在するから,全単射 \(\colon X \to Y\)
が存在する. 従って \(\mathfrak m = \mathfrak n\).最後に,\(X, Y, Z\)
は集合で
\(\mathfrak m = \mathrm{card}(X),\ \mathfrak n = \mathrm{card}(Y),\ \mathfrak l = \mathrm{card}(Z)\)
とおき,\(\mathfrak m \leqslant\mathfrak n\) かつ
\(\mathfrak n \leqslant\mathfrak l\) とする.このとき2つの単射
\(f \colon X \to Y\) および \(g \colon Y \to Z\)
が存在する.\(gf \colon X \to Z\) は単射である.実際,\(gf(x) = gf(x')\)
ならば \(f(x) = f(x')\),従って \(x = x'\).よって \(X\) から \(Z\)
への単射が存在するから \(\mathfrak m \leqslant\mathfrak l\). 以上で
\(\leqslant\) が順序関係であることを示せた.\(\Box\)

\hypertarget{ux547dux984c}{%
\paragraph{命題}\label{ux547dux984c}}
\addcontentsline{toc}{paragraph}{命題}

\(a, b \in \boldsymbol R,\ a \lt b\) とする. \begin{align}
&\mathrm{card}(\boldsymbol Z) = \mathrm{card}(\boldsymbol Q) = \aleph_0,\\
&\mathrm{card}([a, b]) = \mathrm{card}((a, b)) = \mathfrak c,\\
&2^{\aleph_0} = \mathfrak c,\\
&2 \leqslant\mathfrak n \Longrightarrow \mathfrak m \lt \mathfrak n^{\mathfrak m}.
\end{align}
ここで,\([a,b] = \{ x \in \boldsymbol R \,:\,a \leqslant x \leqslant b \}\)
を\textbf{閉区間}(\emph{closed
interval})といい,\((a, b) = \{ x \in \boldsymbol R \,:\,a \lt x \lt b \}\)
を\textbf{開区間}(\emph{open interval})という.

ここから特に,\(\aleph_0 \lt \mathfrak c\) が分かる.

\hypertarget{ux8a3cux660e-2}{%
\paragraph{証明}\label{ux8a3cux660e-2}}
\addcontentsline{toc}{paragraph}{証明}

\hypertarget{ux306eux8a3cux660e}{%
\subparagraph{(1)の証明}\label{ux306eux8a3cux660e}}
\addcontentsline{toc}{subparagraph}{(1)の証明}

\(n \in \boldsymbol N\)
に対して,\(\phi(2n - 1) = -n,\ \phi(2n) = n - 1\)
とすれば,\(\phi \colon \boldsymbol N \to \boldsymbol Z\)
は全単射となる.よって
\(\mathrm{card}(\boldsymbol Z) = \aleph_0\).\(\boldsymbol N\) から
\(\boldsymbol Q\) への全単射を式で表すのは難しい.有理数列 \(\{ a_n \}\)
を次の数列とする: \[
0, 1, \frac 12,  \frac 13, \frac 23, \frac 34, \frac 14, \frac 15, \frac 25, \frac 35, \frac 45, \frac 56, \frac 16, \frac 17, \frac 27, \dotsc
\] 具体的には以下の手順で求める:

\begin{enumerate}
\def\labelenumi{\arabic{enumi}.}
\tightlist
\item
  \(n/m\ (m, n \in \boldsymbol N,\ 1 \leqslant n \leqslant m)\)
  の形の有理数を,\(1 / (2q - 1), \dotsc, (2q - 1) / (2q - 1), 2q / 2q, \dotsc, 1 / 2q\)
  の順番で並べる.
\item
  並べた有理数を順番に見ていき,約分した結果すでに現れた有理数と等しくなるものは除外する.
\item
  残った有理数を順番に番号付ける.
\end{enumerate}

こうして \(\{ a_n \}\)
を得る.この構成から分かるように,\(\boldsymbol Q' = \{ x \in \boldsymbol Q \,:\,0 \leqslant x \leqslant 1 \}\)
とおくと,写像 \(n \mapsto a_n\) は \(\boldsymbol N\) から
\(\boldsymbol Q'\) への全単射になる.あとは \(\boldsymbol Q'\) と
\(\boldsymbol Q\) が対等であることを示せばよい.\(\boldsymbol Q'\) は
\(\boldsymbol Q\) の部分集合である \(\boldsymbol Q'\)
自身と対等であるのは明らかだから,\(\boldsymbol Q\) から
\(\boldsymbol Q'' = \{ x \in \boldsymbol Q \,:\,0 \lt x \lt 1 \} \subset \boldsymbol Q'\)
への全単射の存在を言う.\(x \in \boldsymbol Q''\) に対して,
\begin{align*}
\psi(x) = \left\{ \begin{array}{ll} \frac{x - 1/2}x = 1 - \frac 1{2x} & (\text{if}\ x \leqslant\frac 12),\\
\frac{x - 1/2}{1 - x} = -1 + \frac 1{2(1 - x)} & (\text{if}\ x \gt \frac 12) \end{array} \right.
\end{align*} とすると,\(\psi \colon \boldsymbol Q'' \to \boldsymbol Q\)
は全単射である.実際,\(x \lt x' \lt 1/2 \lt y \lt y'\) のとき
\(\psi(x) \lt \psi(x') \lt \psi(1/2) = 0 \lt \psi(y) \lt \psi(y')\)
だから,\(\psi\) は単射.また,任意に \(z \in \boldsymbol Q\)
を取ると,\(z \gt 0\) ならば \[
\frac 12 \lt \frac{2z + 1}{2(z + 1)} \lt 1,\quad \psi\left( \frac{2z + 1}{2(z + 1)} \right) = z
\] となり,\(z \lt 0\) ならば \[
0 \lt \frac 1{2(1 - z)} \lt \frac 12,\quad \psi\left( \frac 1{2(1 - z)} \right) = z
\] となるから,\(\psi\) は全射でもある.よって \(\boldsymbol Q''\) と
\(\boldsymbol Q\) は対等であり,従って
\(\mathrm{card}(\boldsymbol Q) = \mathrm{card}(\boldsymbol Q') = \aleph_0\).

\hypertarget{ux306eux8a3cux660e-1}{%
\subparagraph{(2)の証明}\label{ux306eux8a3cux660e-1}}
\addcontentsline{toc}{subparagraph}{(2)の証明}

まず \(\mathrm{card}((a, b)) = \mathfrak c\) は, \[
f(x) = \tan\left( \frac \pi2 \cdot \frac{2x - (a + b)}{b - a} \right)
\] によって定まる写像 \(f \colon (a, b) \to \boldsymbol R\)
が全単射であることから分かる.次に
\(\mathrm{card}([a, b]) = \mathfrak c\) を考える.\(\boldsymbol R\) と
\((a, b) \subset [a, b]\) が対等であることは示した.当然 \([a, b]\) は
\([a, b] \subset \boldsymbol R\) と対等だから,結局 \([a, b]\) と
\(\boldsymbol R\) は対等で,\(\mathrm{card}([a, b]) = \mathfrak c\).

\hypertarget{ux306eux8a3cux660e-2}{%
\subparagraph{(3)の証明}\label{ux306eux8a3cux660e-2}}
\addcontentsline{toc}{subparagraph}{(3)の証明}

\(2^{\boldsymbol N}\) から開区間 \((0, 1)\)
への単射と全射をそれぞれ構成する.以下,簡単のため \(2 = \{ 0, 1 \}\)
と書く.写像 \(f \colon \boldsymbol N \to 2\) に対して \begin{align*}
B_f^{(n)} &= \sum_{k = 1}^n \frac{f_k}{2^k},&\quad B_f &= \sum_{k = 1}^\infty \frac{f_k}{2^k},\\
T_f^{(n)} &= \sum_{k = 1}^n \frac{f_k}{3^k},&\quad T_f &= \sum_{k = 1}^\infty \frac{f_k}{3^k},\\
\end{align*} とおき, \begin{align*}
\beta(f) &= \left\{ \begin{array}{ll} B_f & (\text{if}\ B_f \in (0, 1)),\\ \frac 12 & (\text{if}\ B_f \notin (0, 1)), \end{array} \right. \\
\tau(f) &= \left\{ \begin{array}{ll} T_f & (\text{if}\ T_f \in (0, 1)),\\ \frac 23 & (\text{if}\ T_f \notin (0, 1)), \end{array} \right. \\
\end{align*} と定義する(\(B\) はbinary,\(T\)
はternaryの頭文字).このとき,\(\beta \colon 2^{\boldsymbol N} \to (0, 1)\)
は全射で \(\tau \colon 2^{\boldsymbol N} \to (0, 1)\)
は単射であることを示す.

任意の実数 \(r \in (0, 1)\) は2進少数展開 \(r = \sum r_k 2^{-k}\)
を持つ.そこで写像 \(R \colon n \mapsto r_n\)
を考えれば,\(\beta(R) = B_R = r\).よって \(\beta\)
は全射.次に,\(f, g \colon \boldsymbol N \to 2\)
であって,\(f \neq g,\ \tau(f) = \tau(g)\)
なるものが存在したとする.もし \(\tau(f) = \tau(g) = 2/3\)
ならば,\(T_f = T_g = 0\) より \(f = g = 0\)
が従うから,\(\tau(f) = \tau(g) \neq 2/3\)(\(0 \leqslant T_f \leqslant 1/3\)
に注意).このとき \(T_f = T_g\).\(f \neq g\)
であるから,\(f_n \neq g_n\) となる \(n \in \boldsymbol N\)
が存在する.そのような \(n\) のうち最小のものを \(m\) とすると, \[
T_f^{(n)} = T_g^{(n)} \ (n \lt m), \quad T_f^{(m)} \neq T_g^{(m)}.
\] そこで,\(f_m = 1 \gt 0 = g_m\) としてもよく,そのとき
\(T_f^{(m)} \gt T_g^{(m)}\) が成立つ.さて,新しい写像
\(f', g' \colon \boldsymbol N \to 2\) を, \begin{align*}
f'_n &= f_n \ (n \leqslant m), &\quad f'_n &= 0 \ (n \gt m),\\
g'_n &= g_n \ (n \leqslant m), &\quad g'_n &= 1 \ (n \gt m)
\end{align*} と定義する.すると,簡単な計算より, \begin{align*}
T_{f'} &= T_f^{(m - 1)} + \frac 1{3^m},\\
T_{g'} &= T_g^{(m - 1)} + \frac 1{2 \cdot 3^m},\\
\end{align*} が分かる.ここから
\(T_g \leqslant T_{g'} \lt T_{f'} \leqslant T_f\)
となるから,\(T_f \neq T_g\).従って \(\tau\) の単射性が示された.

以上より \(2^{\aleph_0} = \mathrm{card}((0, 1)) = \mathfrak c\) となる.

\hypertarget{ux306eux8a3cux660e-3}{%
\subparagraph{(4)の証明}\label{ux306eux8a3cux660e-3}}
\addcontentsline{toc}{subparagraph}{(4)の証明}

この証明は\textbf{Cantorの対角線論法}(\emph{Cantor's diagonal
argument})とよばれる.\(X, Y\)
を集合として,\(\mathfrak m = \mathrm{card}(X),\ \mathfrak n = \mathrm{card}(Y)\)
とおく.\(\mathfrak n \geqslant 2\) とする.単射 \(\colon X \to Y^X\)
の存在はすぐに分かる.たとえば \(Y\) の異なる2元 \(a, b \in Y\)
を選び(\(\mathfrak n \geqslant 2\)
だから選ぶことができる),\(x \in X\) に対して
\(\chi_x(x) = a\),\(y \neq x\) のとき \(\chi_x(y) = b\) とすれば,写像
\(x \mapsto \chi_x\) が単射である. 全射 \(\colon X \to Y^X\)
が存在しないことを示すために,\(f \colon X \to Y^X\) を全射とする.

ここで,写像 \(g \colon X \to Y\) を,\(f(x)(x) = a\) のとき
\(g(x) = b\),\(f(x)(x) \neq a\) のとき \(g(x) = a\) と定義すれば
\(g \notin f(X)\) である.実際各 \(x \in X\) に対して
\(g(x) \neq f(x)(x)\) となり,\(g = f(y)\) なる \(y \in X\)
は存在しない.これは \(f\) が全射でないことを意味するから矛盾.よって
\(X\) から \(Y^X\) への全射は存在せず,従って
\(\mathfrak m \lt \mathfrak n^{\mathfrak m}\).\(\Box\)

\end{document}
